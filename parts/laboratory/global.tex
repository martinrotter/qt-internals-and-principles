\chapter{Global Qt functions and macros}
Qt has its functionality separated into modules\index{module}. There is one special module (incarnated in QtCore module) called QtGlobal\index{QtGlobal}. QtGlobal is contained within single header file\fdocinlinecode{text}{!}{qglobal.h} and is the part of QtCore library file. QtGlobal contains following:
\begin{itemize}
\item
type clones (\autoref{table:types}) for every standard \cpp{} type,
\item
functions,
\item
macros.\index{macro}
\end{itemize}

\begin{table}[ht]
\begin{center}
\caption{\cpp{} type aliases in Qt}\label{table:types}
\begin{tabular}{c | c}
orignal type name & Qt type name \\
\hline
\fdocinlinecode{cpp}{!}{signed char} & \fdocinlinecode{cpp}{!}{qint8} \\ 
\fdocinlinecode{cpp}{!}{unsigned char} & \fdocinlinecode{cpp}{!}{quint8} \\ 
\fdocinlinecode{cpp}{!}{short} & \fdocinlinecode{cpp}{!}{qint16} \\ 
\fdocinlinecode{cpp}{!}{unsigned short} & \fdocinlinecode{cpp}{!}{quint16} \\ 
\fdocinlinecode{cpp}{!}{int} & \fdocinlinecode{cpp}{!}{qint32} \\ 
\fdocinlinecode{cpp}{!}{unsigned int} & \fdocinlinecode{cpp}{!}{quint32} \\ 
\fdocinlinecode{cpp}{!}{qint64} & \fdocinlinecode{cpp}{!}{qlonglong} \\ 
\fdocinlinecode{cpp}{!}{quint64} & \fdocinlinecode{cpp}{!}{qulonglong} \\ 
\fdocinlinecode{cpp}{!}{unsigned char} & \fdocinlinecode{cpp}{!}{uchar} \\ 
\fdocinlinecode{cpp}{!}{unsigned short} & \fdocinlinecode{cpp}{!}{ushort} \\ 
\fdocinlinecode{cpp}{!}{unsigned int} & \fdocinlinecode{cpp}{!}{uint} \\ 
\fdocinlinecode{cpp}{!}{unsigned long} & \fdocinlinecode{cpp}{!}{ulong} \\
\fdocinlinecode{cpp}{!}{double} & \fdocinlinecode{cpp}{!}{qreal}
\end{tabular}
\end{center}
\end{table}

\section{Fundamental functions}
QtGlobal offers very fundamental functions for value-based comparing and other basic tasks. These functions (often) wrap similar functions from standard C/\cpp{} library\index{standard library}. Most used functions are:
\begin{description}
\item[T qAbs(const T \& value)] \hfill \\
Returns absolute value of input parameter.
\item[const T \& qBound(const T \& min, const T \& value, const T \& max)] \hfill \\
Returns input value \enquote{rounded} to fit within bounds.
\item[double qInf()] \hfill \\
Returns value which represents infinity.
\item[qint64 qRound64(qreal value) and int qRound(qreal value)] \hfill \\
Mathematically rounds input paramater either to 64/32 bit integer.
\end{description}

All functons can be found in\fdocinlinecode{text}{!}{/qt-root-directory/include/QtCore/qglobal.h}. Some other functions will be used throughout the rest of the book.

\section{Producing console outputs with Qt}
Qt offers better way to produce console\index{console} printing for debugging purposes via \fdocinlinecode{cpp}{!}{QDebug}\index{QDebug} class and\fdocinlinecode{cpp}{!}{qInstallMessageHandler} function. You can always use traditional\fdocinlinecode{cpp}{!}{std::cout} for console printing but\fdocinlinecode{cpp}{!}{QDebug} way is much better. Basic syntax for using\fdocinlinecode{cpp}{!}{QDebug} is fairly simple (\autoref{listing:qdebug}) as it implements \fdocinlinecode{cpp}{!}{<<} operator and can act as \fdocinlinecode{cpp}{!}{printf(...)} function.

\begin{fdoccode}{cpp}{listing:qdebug}{Basic QDebug usage}
qDebug() << "Print number " << 10 << ".\n";
qDebug("Print number %d.\n", 10);
\end{fdoccode}

First\fdocinlinecode{cpp}{!}{qDebug} usage requires explicit\fdocinlinecode{cpp}{!}{<QDebug>} inclusion. Second usage acts as wrapper for the\fdocinlinecode{cpp}{!}{printf} function from standard C library. You can use also\fdocinlinecode{cpp}{!}{qWarning},\fdocinlinecode{cpp}{!}{qCritical} or\fdocinlinecode{cpp}{!}{qFatal} functions according to importance of message.

Default implementation halts an application if\fdocinlinecode{cpp}{!}{qFatal} is called and uses\fdocinlinecode{cpp}{!}{std::cerr} output for printing messages. You can implement custom behavior for previous functions very simply:
\begin{enumerate}
\item
You need to implement global function (or static method) with signature\fdocinlinecode{cpp}{!}{void (*function)(QtMsgType, const QMessageLogContext &, const QString &)}. Typical\\ implementation may look like \autoref{listing:qdebug2}.

\item
You need to assign handler to this function via\fdocinlinecode{cpp}{!}{qInstallMessageHandler} function.
\end{enumerate}

\begin{fdoccode}{cpp}{listing:qdebug2}{Typical printing handler for QDebug}
void debug_handler(QtMsgType type, const QMessageLogContext &placement, const QString &message) {
    switch (type) {
	case QtDebugMsg:
	    fprintf(stderr, "[%s] INFO (%s, line %d) : %s\n",
		    APP_LOW_NAME,
		    placement.file,
		    placement.line,
		    qPrintable(message));
	    break;
	case QtWarningMsg:
	    fprintf(stderr, "[%s] WARNING (%s, line %d) : %s\n",
		    APP_LOW_NAME,
		    placement.file,
		    placement.line,
		    qPrintable(message));
	    break;
	case QtCriticalMsg:
	    fprintf(stderr, "[%s] CRITICAL (%s, line %d) : %s\n",
		    APP_LOW_NAME,
		    placement.file,
		    placement.line,
		    qPrintable(message));
	    break;
	case QtFatalMsg:
	    fprintf(stderr, "[%s] FATAL (%s, line %d) : %s\nApplication is halting now.\n",
		    APP_LOW_NAME,
		    placement.file,
		    placement.line,
		    qPrintable(message));
	    qApp->exit(EXIT_FAILURE);
    }
}
\end{fdoccode}

Calling\fdocinlinecode{cpp}{!}{qFatal} function results in error dialog in Windows operating system (\autoref{figure:errordialog}).

\begin{figure}[ht]
\centering\includegraphics[width=9cm]{graphics/laboratory/11-debugdialog.png}
\caption{Application crash dialog in Windows}\label{figure:errordialog}
\end{figure}

\begin{fdocextra}
Debugging outputs should always be written in English language with ASCII character set.
\end{fdocextra}

This implementation prints out extra information which is extremely important for debugging. However, you can do whatever you want in your implementation. Storing outputs to database or sending them over network are just some possible enhancements.

There is another (simpler) way of forcing Qt to format console outputs. You may tweak\fdocinlinecode{text}{!}{QT_MESSAGE_PATTERN} build environment variable to display application name or other useful information. Head to \citep{various:qtdoc} for more information or browse source code in\fdocinlinecode{text}{!}{sources/laboratory/07-qdebug}.